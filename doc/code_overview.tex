\documentclass[11pt]{article}
\textheight 10in
\topmargin -0.5in
\textwidth 6.5in
\oddsidemargin -0.2in
\parindent=0.3in

\usepackage{epsfig}
% Defines paragraphs
\setlength{\parskip}{\bigskipamount}
\setlength{\parindent}{0em}

\pagestyle{myheadings}
\markright{GLIMMER code design overview}

\begin{document}

\title{GLIMMER code overview}
\date{April 2004}
\author{Ian Rutt}
\maketitle

\section{Introduction}

The `design' of GLIMMER is a consequence of the way it has been
developed. Initially, as a stand-alone model with a single domain, module
variables were used to hold all model fields and parameters. With the move to
use GLIMMER as the ice model component within GENIE, and the desire to enable
several active regions to be run simultaneously, the module variables were
converted into components of derived types, and an extra layer added on top of
the exisiting structure to deal with global fields and parameters, and deal
with the downscaling/interpolation of input fields. The result is a structure
that is probably more complex than it needs to be, but still hopefully
reasonably logical.

\section{Data structure}

The derived type \texttt{glimmer\_params} is the top-level data-structure in
GLIMMER. It contains the global parameters for the model, as well as the
global fields used in the temporal averaging of input fields. Its primary
function, however, is to hold an array of ice model instances (type
\texttt{glimmer\_instance}). This array is allocated at run-time. Each ice
model instance contains instances of derived types (\texttt{projection} and
\texttt{downscale}) concerning the relationship between the local and global
model grids, as well as upscaling parameters (which should have their own
derived type, really), and an instance of \texttt{glimmer\_global\_type}. This
last derived type contains single instances of eighteen other derived types,
which were replacements for the variable-containing modules of the original
ice model. It is debatable whether \texttt{glimmer\_global\_type} is strictly necessary,
and it might be worth reorganising the whole structure at some point, though
this would obviously be a big job. The situation is summarised in figure
\ref{main_class_diagram}.

\begin{figure}
\centering
\epsfig{figure=class_diagram.eps,width=\textwidth}
\caption{Main `Class Diagram' for GLIMMER. The relationship between the top-level
  \texttt{glimmer\_params} type and its component types is shown. The
  components of the \texttt{glimmer\_global\_type} type are formed from the
  modules of the original ice model.}
\label{main_class_diagram}
\end{figure}

\end{document}