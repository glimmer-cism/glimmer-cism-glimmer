\section{Ice Thickness Evolution}
The evolution of the ice thickness, $H$, stems from the continuity equation and can be expressed as
\begin{equation}
  \label{kin.eq.ice_thickness}
  \frac{\pd H}{\pd t} = -\vec\nabla\cdot(\overline{\vec{u}} H) + b,
\end{equation}
where $\overline{\vec{u}}$ is the vertically averaged ice velocity, $b$ is the surface mass balance and $\vec\nabla$ is the horizontal gradient operator. 

For large--scale ice sheet models, the \emph{shallow ice approximation} is generally used. This approximation states that bedrock and ice surface slopes are assumed sufficiently small so that the normal stress components can be neglected. The horizontal shear stresses ($\tau_{xz}$ and $\tau_{yz}$) can thus be approximated by
\begin{equation}
  \label{kin.eq.horiz_shear}
  \begin{split}
    \tau_{xz}(z)&=-\rho g(s-z)\frac{\pd s}{\pd x},\\
    \tau_{yz}(z)&=-\rho g(s-z)\frac{\pd s}{\pd y},
  \end{split}
\end{equation}
where $\rho$ is the density of ice, $g$ the acceleration due to gravity and $s=H+h$ the ice surface.

Strain rates $\dot{\epsilon}_{ij}$ of polycrystalline ice are related to the stress tensor by the non--linear flow law:
\begin{equation}
  \label{kin.eq.flowlaw}
  \dot{\epsilon}_{iz}=\frac12\left(\frac{\pd u_i}{\pd z}+\frac{\pd u_z}{\pd i}\right)=A(T^\ast)\tau_\ast^{(n-1)}\tau_{iz}\qquad i=x,y,
\end{equation}
where $\tau_\ast$ is the effective shear stress defined by the second invariant of the stress tensor, $n$ the flow law exponent and $A$ the temperature--dependent flow law coefficient. $T^\ast$ is the absolute temperature corrected for the dependence of the melting point on pressure. The parameters $A$ and $n$ have to
 be found by experiment. $n$ is usually taken to be 3. $A$ depends on factors such as temperature, crystal size and orientation, and ice impurities. Experiments suggest that $A$ follows the Arrhenius relationship:
\begin{equation}
  \label{kin.eq.arrhenius}
  A(T^\ast)=fae^{-Q/RT^\ast},
\end{equation}where $a$ is a temperature--independent material constant, $Q$ is the activation energy for creep and $R$ is the universal gas constant. $f$ is a tuning parameter used to `speed--up' ice flow and accounts for ice impurities and the development of anisotropic ice fabrics.

Integrating \eqref{kin.eq.arrhenius} with respect to $z$ gives the horizontal velocity profile:
\begin{equation}
  \label{kin.eq.horiz_velo}
  \vec u(z)-\vec u(h) = -2(\rho g)^n|\vec\nabla s|^{n-1}\vec\nabla s\int_h^zA(s-z)^ndz,
\end{equation}
where $\vec u(h)$ is the basal velocity (sliding velocity). Integrating \eqref{kin.eq.horiz_velo} again with respect to $z$ gives an expression for the vertically averaged ice velocity:
\begin{equation}
  \label{kin.eq.avg_velo}
  \overline{\vec u}H=-2(\rho g)^n|\vec\nabla s|^{n-1}\vec\nabla s\int_h^s\int_h^zA(s-z)^ndzdz'.
\end{equation}

The vertical ice velocity stems from the conservation of mass for an incompressible material:
\begin{equation}
  \label{kin.eq.incompress}
  \frac{\pd u_x}{\pd x} + \frac{\pd u_y}{\pd y} + \frac{\pd u_z}{\pd z} = 0.
\end{equation}
Integrating \eqref{kin.eq.incompress} with respect to $z$ gives the vertical velocity distribution of each ice column:
\begin{equation}
  \label{kin.eq.vert_velo}
  w(z)=-\int_h^z\vec\nabla\cdot\vec u(z)dz+w(h),
\end{equation}
with lower, kinematic boundary condition
\begin{equation}
  w(h)=\frac{\pd h}{\pd t}+\vec u(h)\cdot\vec\nabla h+m,
\end{equation}
where $m$ is the melt rate at the ice base. The upper kinematic boundary is given by the surface mass balance and must satisfy:
\begin{equation}
  \label{kin.eq.upper_bc}
  w(s)=\frac{\pd s}{\pd t}+\vec u(s)\cdot\vec\nabla s+b.
\end{equation}

\input{\dir/grid.tex}

\subsection{Ice Sheet Equations in $\sigma$--Coordinates}
The horizontal velocity, Equation \eqref{kin.eq.horiz_velo}, becomes in the $\sigma$--coordinate system
\begin{equation}
  \label{kin.eq.vert_velo_sigma}
  \vec u(\sigma) = -2(\rho g)^nH^{n+1}|\vec\nabla s|^{n-1}\vec\nabla s\int_1^\sigma A\sigma^nd\sigma+\vec u(1)
\end{equation}
and the vertically averaged velocity
\begin{equation}
  \label{kin.eq.avg_velo_scaled}
  \overline{\vec u} H=H\int_0^1\vec ud\sigma+\vec u(1)H
\end{equation}
The vertical velocity, Equation \eqref{kin.eq.vert_velo}, becomes
\begin{equation}
  \label{kin.eq.vert_velo_scaled}
  w(\sigma)=-\int_1^\sigma\left(\frac{\pd\vec u}{\pd\sigma}\cdot(\vec\nabla s-\sigma\vec\nabla H)+H\vec\nabla\cdot\vec u\right)d\sigma+w(1)
\end{equation}
and lower boundary condition
\begin{equation}
  w(1)=\frac{\pd h}{\pd t}+\vec u(1)\cdot\vec\nabla h+m.
\end{equation}

\subsection{Semi--Implicit Scheme for the Ice Thickness Equation}
Equation \eqref{kin.eq.ice_thickness} can be rewritten as a diffusion equation, with non--linear diffusion coefficient $D$:
\begin{equation}
  \label{kin.eq.ice_evo}
  \frac{\pd H}{\pd t}=-\vec\nabla\cdot D\vec\nabla s+b=\vec\nabla\cdot\vec q+a
\end{equation}
This non--linear partial differential equation can be linearised by using the diffusion coefficient from the previous time step. The diffusion coefficient is calculated on the $(r,s)$--grid, i.e. staggered in both $x$ and $y$ direction. Using finite differences, the fluxes in $x$ direction, $q^x$ become
\begin{subequations}
\begin{align}
  q^x_{i+\frac12,j}&=-\frac12(\tilde{D}_{r,s}+\tilde{D}_{r,s-1})\frac{s_{i+1,j}-s_{i,j}}{\Delta x}\\
  q^x_{i-\frac12,j}&=-\frac12(\tilde{D}_{r-1,s}+\tilde{D}_{r-1,s-1})\frac{s_{i,j}-s_{i-1,j}}{\Delta x}\\
  \intertext{and the fluxes in $y$ direction}
  q^y_{i,j+\frac12}&=-\frac12(\tilde{D}_{r,s}+\tilde{D}_{r-1,s})\frac{s_{i,j+1}-s_{i,j}}{\Delta y}\\
  q^y_{i,j-\frac12}&=-\frac12(\tilde{D}_{r,s-1}+\tilde{D}_{r-1,s-1})\frac{s_{i,j}-s_{i,j-1}}{\Delta y}
\end{align}  
\end{subequations}

The semi--implicit temporal discretisation of \eqref{kin.eq.ice_evo} is then:
\begin{equation}
\label{kin.eq.ice_evo_disc1}
  \frac{H^{t+1}_{i,j}-H^t_{i,j}}{\Delta t}=\frac{q^{x,t+1}_{i+\frac12,j}-q^{x,t+1}_{i-\frac12,j}}{\Delta x}+\frac{q^{y,t+1}_{i,j+\frac12}-q^{y,t+1}_{i,j-\frac12}}{\Delta y} + a_{i,j}
\end{equation}
The superscripts $^t$ and $^{t+1}$ indicate at what time the ice thickness $H$ is evaluated. Collecting all $H^{t+1}$ terms of \eqref{kin.eq.ice_evo_disc1} on the LHS and moving all other terms to the RHS we can rewrite \eqref{kin.eq.ice_evo_disc1} as
\begin{multline}
  -\alpha^1_{i,j}H^{t+1}_{i-1,j} - \alpha^2_{i,j}H^{t+1}_{i+1,j} - \alpha^3_{i,j}H^{t+1}_{i,j-1} - \alpha^4_{i,j}H^{t+1}_{i,j+1}+ (1-\alpha^5_{i,j})H^{t+1}_{i,j} = \\
  (\alpha^1_{i,j}h_{i-1,j} + \alpha^2_{i,j}h_{i+1,j} + \alpha^3_{i,j}h_{i,j-1} + \alpha^4_{i,j}h_{i,j+1}+ \alpha^5_{i,j}h_{i,j}) + a_{i,j}\Delta t
\end{multline}
with the elements of the sparse matrix
\begin{subequations}
  \begin{align}
    \alpha^1_{i,j} &=\frac{\tilde{D}_{r-1,s}+\tilde{D}_{r-1,s-1}}{2\Delta x^2}\Delta t\\
    \alpha^2_{i,j} &=\frac{\tilde{D}_{r,s}+\tilde{D}_{r,s-1}}{2\Delta x^2}\Delta t\\
    \alpha^3_{i,j} &=\frac{\tilde{D}_{r,s-1}+\tilde{D}_{r-1,s-1}}{2\Delta y^2}\Delta t\\
    \alpha^4_{i,j} &=\frac{\tilde{D}_{r,s}+\tilde{D}_{r-1,s}}{2\Delta y^2}\Delta t\\
    \alpha^5_{i,j} &=-(\alpha^1_{i,j}+\alpha^2_{i,j}+\alpha^3_{i,j}+\alpha^4_{i,j})
  \end{align}
\end{subequations}

This matrix equation is solved using an iterative matrix solver for non-symmetric sparse matrices. The solver used here is the bi--conjugate gradient method with incomplete LU decomposition preconditioning provided by the SLAP package.

\subsubsection{Calculating the Horizontal Velocity and the Diffusivity}
Horizontal velocity and diffusivity calculations are split up into two parts:
\begin{subequations}
  \begin{align}
    \vec u(\sigma)&=c\vec\nabla s+\vec u(1)\\
    D &=H\int_0^1cd\sigma\\
    \vec q&=D\vec\nabla s+H\vec u(1)\\
    \intertext{with}
    c(\sigma)&=-2(\rho g)^nH^{n+1}|\vec\nabla s|^{n-1}\int_1^\sigma A\sigma^nd\sigma
  \end{align}
\end{subequations}

Quantities $\vec u$ and $D$ are found on the velocity grid. Integrating from the ice base ($k=N-1$), the discretised quantities become
\begin{subequations}
  \begin{align}
    \tilde{c}_{r,s,N}&=0\\
    \tilde{c}_{r,s,k}&=-2(\rho g)^nH_{r,s}^{n+1}\left(({\tilde{s}^x_{r,s}})^2+({\tilde{s}^y_{r,s}})^2\right)^{\frac{n-1}{2}}\sum_{\kappa=N-1}^k\frac{A_{r,s,\kappa}+A_{r,s,\kappa+1}}2 \left(\frac{\sigma_{\kappa+1}+\sigma_\kappa}2\right)^n(\sigma_{\kappa+1}-\sigma_\kappa)\\
    \tilde{D}_{r,s}&=H_{r,s}\sum_{k=0}^{N-1}\frac{\tilde{c}_{r,s,k}+\tilde{c}_{r,s,k+1}}2(\sigma_{k+1}-\sigma_k)
  \end{align}      
\end{subequations}
Expressions for $\vec{u}_{i,j,k}$ and $\vec{q}_{i,j}$ are straight forward.

\input{\dir/vert_velo.tex}