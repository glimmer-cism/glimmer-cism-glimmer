\section{Getting and Installing GLIMMER}
GLIMMER is a relatively complex system of libraries and programs which build on other libraries. This section documents how to get GLIMMER and its prerequisites, compile and install it. Please report problems and bugs to the \href{http://forge.nesc.ac.uk/mailman/listinfo/glimmer-discuss}{GLIMMER mailing list}.

\subsection{Prerequisites}
GLIMMER is distributed as source code, a sane built environment is therefore required to compile the system. On UNIX systems \href{http://www.gnu.org/software/make/}{GNU make} is suggested since the Makefiles may rely on some GNU make specific features. There are two ways of getting the source code:
\begin{enumerate}
\item download a released version from the GLIMMER website, or
\item download the latest developer's version of GLIMMER and friends from \href{http://forge.nesc.ac.uk/}{NeSCForge} using \href{http://www.gnu.org/software/cvs/}{CVS}.
\end{enumerate}
If you want to build GLIMMER from CVS then you need GNU autoconf and automake to generate the built system and \href{http://www.python.org}{python} which is used for analysing dependencies and f95 code generation. Furthermore, the Python scripts rely on language features which were only introduced with version 2.3.

GLIMMER is mostly written in FORTRAN95, a good f95 compiler is, therefore, required. GLIMMER is known to work with the NAGware f95, Intel ifort compilers. GLIMMER does not compile with the SUN WS 7.0 f95 compiler due to a compiler bug. The current SUN f95 compiler might work but has not been tested yet.

In addition to f90 and python you will also need {\LaTeX} if you want to build the documentation.

Furthermore, GLIMMER depends on these packages:
\begin{itemize}
\item \href{http://www.unidata.ucar.edu/packages/netcdf/index.html}{{\bf netCDF}}: You will most likely need to compile and install the netCDF library yourslef since the packages usually do not contain the f90 bindings which are used by GLIMMER.
\item \href{http://www.remotesensing.org/proj}{{\bf PROJ.4}}: The cartographic projections library which is used to handle projections.
\item \href{http://gmt.soest.hawaii.edu/}{{\bf GMT}}: The Generic Mapping Tools are used for visualisation.
\end{itemize}

\subsection{Installing a Released Version of GLIMMER}\label{ug.sec.tarball}
We recommend that GLIMMER and its components is installed in its own directory tree, e.g. \texttt{GLIMMER}, and sources in a subdirectory, e.g. \texttt{GLIMMER/src}. 

Once you have downloaded the tar-balls from the GLIMMER website you need to unpack them using
\begin{verbatim}
tar -xvzf libfproj4-VERS.tar.gz
\end{verbatim}
and
\begin{verbatim}
tar -xvzf glimmer-VERS.tar.gz
\end{verbatim}
where \texttt{VERS} is the package version.

The packages are then compiled using the usual GNU sequence of commands:
\begin{verbatim}
./configure [options]
make
make install
\end{verbatim}
The options and relevant environment variables are described in Table \ref{ug.tab.env}.
\begin{table}[htbp]
  \centering
  \begin{tabular}{|l|p{10cm}|}
    \hline
    Variable & Description \\
    \hline
    \texttt{FC} & f95 compiler to be used \\
    \texttt{FCFLAGS} & flags passed to the f95 compiler \\
    \texttt{LDFLAGS} & linker flags\\
    \hline
    Option  & Description \\
    \hline
    \texttt{--help} & print help \\
    \texttt{--prefix} & the installation prefix, e.g. \texttt{GLIMMER} \\
    \texttt{--with-netcdf} & prefix where the netCDF library is installed \\
    \texttt{--with-blas} & extra libraries used to provide BLAS functionality. A built--in, non--optimised version of BLAS is used if this option is not used. \\
    \texttt{--enable-doc} & build documentation.\\
    \texttt{--enable-profile} & enable profiling of GLIMMER (see Sec. \ref{ug.sec.profile})\\
    \hline
  \end{tabular}
  \caption{Environment variables and \texttt{configure} options used by GLIMMER.}
  \label{ug.tab.env}
\end{table}

\subsection{Installing from CVS}
Revisions of GLIMMER are managed using CVS. You can download the latest development version of GLIMMER and proj4 using the following sequence of cvs commands:
\begin{verbatim}
cvs -d:pserver:anonymous@forge.nesc.ac.uk:/cvsroot/glimmer login
cvs -z3 -d:pserver:anonymous@forge.nesc.ac.uk:/cvsroot/glimmer co proj4
cvs -z3 -d:pserver:anonymous@forge.nesc.ac.uk:/cvsroot/glimmer co glimmer
\end{verbatim}

The cvs version does not include some automatically generated files. In order to be able to compile the cvs version you need the GNU autotools and python. The built scripts are generated by running
\begin{verbatim}
./bootstrap
\end{verbatim}
in the \texttt{proj4} and \texttt{glimmer} source directories. The packages are then configured and built as described in Section \ref{ug.sec.tarball}.


\subsection{Profiling}\label{ug.sec.profile}
You can enable profiling by setting the environment variable \texttt{FFLAGS} to contain \texttt{-DPROFILING}. By default times are integrated over 100 time steps. You can cheange this behaviour by setting the variable \texttt{PROFILE\_PERIOD}. The timing data is written to the file \texttt{glide.profile} which contains 5 columns of data (see Table \ref{ug.tab.profile_format}).
\begin{table}[htbp]
  \centering
  \begin{tabular}{|l|l|}
    \hline
    Column 1 &total CPU time elapsed when data is written to file\\
    Column 2 &accumulated time spent on this block of calculations\\
    Column 3 &integer ID used to identify this block of calculations\\
    Column 4 &model year\\
    Column 5 &description of this block of calculations\\
    \hline
  \end{tabular}
  \caption{File format of profile data file.}
  \label{ug.tab.profile_format}
\end{table}
A python script using the PyGMT library to visualise the profile is provided.

\subsection{GMT Visualisation Tools}
The visualisation tools are based on GMT programs written in Python. In order to be able to use these it is necessary to have a few extra packages:
\begin{itemize}
\item \href{http://www.pfdubois.com/numpy/}{{\bf Numerical Python}}: Numerical Python extends python to be able to work on gridded data.
\item \href{http://starship.python.net/~hinsen/ScientificPython/}{{\bf Scientific Python}}: Scientific Python provides access to netCDF files. Make sure you enable the netCDF specific modules.
\end{itemize}

The visualisation tools are based on \texttt{PyGMT} which you can get from NeSCForge
{\small
\begin{verbatim}
cvs -d:pserver:anonymous@forge.nesc.ac.uk:/cvsroot/pygmt login
cvs -z3 -d:pserver:anonymous@forge.nesc.ac.uk:/cvsroot/pygmt co PyGMT
\end{verbatim}}
\texttt{PyGMT} is installed using
{\small
\begin{verbatim}
python setup.py install --home=$GLIMMER_PREFIX
\end{verbatim}}

The visualisation tools are part of the GLIMMER CVS repository. You can get them using
{\small
\begin{verbatim}
cvs -d:pserver:anonymous@forge.nesc.ac.uk:/cvsroot/glimmer login
cvs -z3 -d:pserver:anonymous@forge.nesc.ac.uk:/cvsroot/glimmer co PyCF
\end{verbatim}}
and install them
{\small
\begin{verbatim}
python setup.py install --home=$GLIMMER_PREFIX
\end{verbatim}}
