\section{GLIMMER subprogram calls}
%
This section details the subroutine calls provided by GLIMMER, and their
arguments. Note that where a type is given as \texttt{real(rk)}, this
indicates that the kind of the real type is specified by the value of
parameter \texttt{rk}, which may be altered in the file \texttt{glimmer\_globals.f90}.
%
\subsection{Subroutine \texttt{initialise\_glimmer}}
%
\paragraph{Purpose} To initialise the ice model, and load in all relevant parameter files.
%
\paragraph{Name and mandatory arguments}
%
\begin{verbatim}
  subroutine initialise_glimmer(params,lats,longs,paramfile)
\end{verbatim}
%
\paragraph{Arguments}
%
\begin{center}
\begin{tabular}{llll}
\texttt{params}    & \texttt{type(glimmer\_params)} & \texttt{intent(inout)} &
Ice model to be configured \\
\texttt{lats(:)}   & \texttt{real(rk)} & \texttt{intent(in)} & latitudinal location of grid-points in \\
 & & & global data (given in $^{\circ}\mathrm{N}$)\\
\texttt{longs(:)}  & \texttt{real(rk)} & \texttt{intent(in)} & longitudinal location of grid-points in \\
 & & & global data (given in $^{\circ}\mathrm{E}$)\\
\texttt{paramfile} & \texttt{character(*)} & \texttt{intent(in)} & name of
top-level GLIMMER \\
 & & & parameter file \\
\end{tabular}
\end{center}
%
\paragraph{Additional notes}
%
\begin{itemize}
\item The ice model determines the size of the global domain from the sizes of
  the arrays \texttt{lats} and \texttt{longs}.
\item The latitudes contained in \texttt{lats} must be in descending order, so
  that $\mathtt{lats(i)}>\mathtt{lats(i+1)}$ for $1\leq \mathtt{i} \leq
  \mathtt{size(lats)}$.
\end{itemize}
%
\subsection{Subroutine \texttt{glimmer}}
%
\paragraph{Purpose}
%
To perform temporal averaging of input fields, and, if necessary, down-scale
those fields onto local projections and perform an ice model time-step. Output
files may be appended to, and if optional arguments used, fields made
available for feedback.
%
\paragraph{Name and mandatory arguments}
%
\begin{verbatim}
  subroutine glimmer(params,time,temp,precip,zonwind,merwind,orog)
\end{verbatim}
%
\paragraph{Mandatory arguments}
%
\begin{center}
\begin{tabular}{llll}
\texttt{params} & \texttt{type(glimmer\_params)} & \texttt{intent(inout)} &
parameters for this run \\
\texttt{time} & \texttt{real(rk)} & \texttt{intent(in)} & Current model time
(hours) \\
\texttt{temp(:,:)} & \texttt{real(rk)} & \texttt{intent(in)} & Daily mean surface
temperature field ($^{\circ}\mathrm{C}$) \\
\texttt{precip(:,:)} & \texttt{real(rk)} & \texttt{intent(in)} & Daily
precipitation \\
 & & & field (mm/day) \\
\texttt{zonwind(:,:)} & \texttt{real(rk)} & \texttt{intent(in)} & Zonal
component of the wind field \\
 & & & ($\mathrm{ms}^{-1}$) \\
\texttt{merwind(:,:)} & \texttt{real(rk)} & \texttt{intent(in)} & Meridional 
component of the wind \\
 & & & field ($\mathrm{ms}^{-1}$) \\
\texttt{orog(:,:)} & \texttt{real(rk)} & \texttt{intent(in)} & Global orography (m) \\
\end{tabular}
\end{center}
%
\paragraph{Optional arguments}
%
\begin{center}
\begin{tabular}{llll}
\texttt{output\_flag} & \texttt{logical} & \texttt{intent(out)} & Set to show
new output fields have \\
 & & & been calculated after an ice-model time-step. \\
 & & & If this flag is not set, the output fields \\
 & & & retain their values at input. \\
\texttt{orog\_out(:,:)} & \texttt{real(rk)} & \texttt{intent(inout)} & Output
orography (m)\\ 
\texttt{albedo(:,:)} & \texttt{real(rk)} & \texttt{intent(inout)} & Surface
albedo \\
\texttt{ice\_frac(:,:)} & \texttt{real(rk)} & \texttt{intent(inout)} &
Fractional ice coverage \\
\texttt{fw\_flux(:,:)} & \texttt{real(rk)} & \texttt{intent(inout)} & The
fresh-water flux (mm/a) \\
 & & & This is simply the ablation calculated by the \\
 & & & model, scaled up to the global grid. It is up \\
 & & & to the global model to then  deal with it \\
 & & & (route it to the oceans, land scheme, etc.) \\
\end{tabular}
\end{center}
\paragraph{Additional notes}
%
\begin{itemize}
\item The sizes of all two-dimensional fields passed as arguments must be the
  same as that implied by the sizes of the arrays used to pass latitude and
  longitude information when the model was initialised using
  \texttt{initialise\_glimmer}. There is
  currently no checking mechanism in place for this, so using fields of the wrong size
  will lead to unpredictable results.
\item Zonal and meridional components of the wind are only required if the
  small-scale precipitation parameterization is being used (with
  \texttt{whichprecip} set to 2). In other circumstances, \texttt{zonwind} and
  \texttt{merwind} must still be arrays of the correct rank, but need not be
  the correct size, or may be unallocated if desired.
\item The optional output fields only refers to the parts of the globe
  covered by the GLIMMER ice model instances. The fraction of each global
  grid-box covered by ice model instances may be obtained using the
  \texttt{glimmer\_coverage\_map} subroutine below. 
\item The output orography field is given as a mean calculated over the part
  of the grid-box covered by ice  model instances. Thus, to calculate the
  grid-box mean, the output fields should be multiplied point-wise by the
  coverage fraction. 
\item Albedo is currently fixed at 0.4 for ice-covered ground, and set to zero
  elsewhere. The albedo is given for the part of the global grid box covered
  by ice, not as an average of the part covered by the ice model. No attempt
  is made to guess the albedo of the parts of the ice model domain \emph{not}
  covered by ice.
\end{itemize}
%
\paragraph{Example interpretation of output fields}
%
Consider a particular point, $(i,j)$ in the global domain. Suppose value
returned by \texttt{glimmer\_coverage\_map} for this point is 0.7, and the
output fields have these values:
\begin{verbatim}
  orog_out(i,j)  = 200.0
  albedo(i,j)    =   0.4
  ice_frac(i,j)  =   0.5
\end{verbatim}
%
What does this mean? Well, the ice model covers 70\% of the grid-box, and in
that part the mean surface elevation is 200\,m. Of the part covered by the ice
model, half is actually covered by ice. Thus, 35\% ($0.5\times 0.7$) of the global grid-box is
covered by ice, and the ice has an mean albedo of 40\%. The model makes no suggestion for the
albedo or elevation of the other 65\% of the grid-box.
%
\subsection{Subroutine \texttt{end\_glimmer}}
%
\paragraph{Purpose} To perform general tidying-up operations, close files, etc.
%
\paragraph{Name and mandatory arguments}
%
\begin{verbatim}
  subroutine end_glimmer(params)
\end{verbatim}
%
\paragraph{Arguments}
%
\begin{center}
\begin{tabular}{llll}
\texttt{params} & \texttt{type(glimmer\_params)} & \texttt{intent(inout)} & Ice model
parameters \\
\end{tabular}
\end{center}
%
\subsection{Function \texttt{glimmer\_coverage\_map}}
%
\paragraph{Purpose} To obtain a map of fractional coverage of global
grid-boxes by GLIMMER ice model instances. The function returns a value
indicating success, or giving error information.
%
\paragraph{Type, name and mandatory arguments}
%
\begin{verbatim}
  integer function glimmer_coverage_map(params,coverage)
\end{verbatim}
%
\paragraph{Arguments}
%
\begin{center}
\begin{tabular}{llll}
\texttt{params} & \texttt{type(glimmer\_params)} & \texttt{intent(in)} & Ice model parameters \\
\texttt{coverage(:,:)} & \texttt{real(rk)} & \texttt{intent(out)} & Coverage
map \\
\end{tabular}
\end{center}
%
\paragraph{Returned value}
%
\begin{center}
\begin{tabular}{|c|l|}
\hline
Value & Meaning \\
\hline
\hline
0 & Coverage map has been returned successfully \\
1 & Coverage map not yet calculated; must call \texttt{initialise\_glimmer}
first \\
2 & Array \texttt{coverage} is the wrong size \\
\hline
\end{tabular}
\end{center}
%
\subsection{Function \texttt{glimmer\_main\_funit}}
%
\paragraph{Purpose}
%
To return the value of the main logical file unit used by glimmer for writing
and reading files. This unit is used for all read/write operations except for
the glimmer log file (\texttt{glimmer.gll}), which uses a fixed unit (21). The
default value for the unit is 20, but this may be changed at any time, even
when the model is being run.
%
\paragraph{Type, name and mandatory arguments}
%
\begin{verbatim}
  integer glimmer_main_funit(params)
\end{verbatim}
%
\paragraph{Arguments}
%
\begin{center}
\begin{tabular}{llll}
\texttt{params} & \texttt{type(glimmer\_params)} & \texttt{intent(in)} & Ice model parameters \\
\end{tabular}
\end{center}
%
\paragraph{Returned value}
%
The returned value is the logical file unit being used.
%
\subsection{Subroutine \texttt{glimmer\_set\_main\_funit}}
%
\paragraph{Purpose}
%
To set the value of the logical file unit being used by glimmer for writing
and reading files. See previous entry for more details.
%
\paragraph{Name and mandatory arguments}
%
\begin{verbatim}
  glimmer_set_main_funit(params,unit)
\end{verbatim}
%
\paragraph{Arguments}
%
\begin{center}
\begin{tabular}{llll}
\texttt{params} & \texttt{type(glimmer\_params)} & \texttt{intent(inout)} &
Ice model parameters \\
\texttt{unit} & \texttt{integer} & \texttt{intent(in)} & Logical file unit to
be set \\
\end{tabular}
\end{center}
%
\subsection{Subroutine \texttt{glimmer\_write\_restart}}
%
\paragraph{Purpose}
%
To write a restart file containing the whole model state, including all ice
model instances and associated projection data.
%
\paragraph{Name and mandatory arguments}
%
\begin{verbatim}
  glimmer_write_restart(params,unit,filename)
\end{verbatim}
%
\paragraph{Arguments}
%
\begin{center}
\begin{tabular}{llll}
\texttt{params} & \texttt{type(glimmer\_params)} & \texttt{intent(in)} &
Ice model parameters \\
\texttt{unit} & \texttt{integer} & \texttt{intent(in)} & Logical file unit to
use \\
\texttt{filename} & \texttt{character(*)} & \texttt{intent(in)} & Filename to
write \\
\end{tabular}
\end{center}
%
\subsection{Subroutine \texttt{glimmer\_read\_restart}}
%
\paragraph{Purpose}
%
To read a restart file containing the whole model state, including all ice
model instances and associated projection data.
%
\paragraph{Name and mandatory arguments}
%
\begin{verbatim}
  glimmer_read_restart(params,unit,filename)
\end{verbatim}
%
\paragraph{Arguments}
%
\begin{center}
\begin{tabular}{llll}
\texttt{params} & \texttt{type(glimmer\_params)} & \texttt{intent(out)} &
Ice model parameters \\
\texttt{unit} & \texttt{integer} & \texttt{intent(in)} & Logical file unit to
use \\
\texttt{filename} & \texttt{character(*)} & \texttt{intent(in)} & Filename to
read \\
\end{tabular}
\end{center}
