\section{Configuration files}
%
This section contains information about the contents of the files
used to configure GLIMMER.
%
\subsection{Top-level GLIMMER configuration file}
%
The top-level configuration file contains only a single section,
\texttt{[parameters]}, which contains two possible entries:
\begin{center}
\begin{tabular}{|c|c|c|l|}
\hline
Name & Type & Default & Description (units) \\
\hline
\hline
\texttt{time-step} & real & 1.0 & Main model time-step (years) \\
\hline
\texttt{instance filenames} & character array & --- & List of
instance-specific filenames \\
\hline
\end{tabular}
\end{center}
%
Note that the number of ice model instances created is determined by the
number of elements in the array of instance-specific filenames given
here. Thus, it is essential to list at least one filename.
%
\subsection{Instance-specific configuration files}
%
The instance-specific configuration files may contain the following sections:
\begin{center}
\begin{tabular}{l}
\texttt{[output]} \\
\texttt{[domain size]} \\
\texttt{[projection]} \\
\texttt{[sigma coordinates]} \\
\texttt{[options]} \\
\texttt{[timesteps]} \\
\texttt{[grid-lengths]} \\
\texttt{[parameters]} \\
\texttt{[forcing]} \\
\texttt{[constants]} \\
\texttt{[PDD scheme]} \\
\end{tabular}
\end{center}
%
\subsubsection{Possible \texttt{[output]} section entries}
%
This section may only have one entry, \texttt{configuration file}, which
gives the name of the NetCDF I/O configuration file for this instance. As
noted above, netCDF I/O is controlled by a separate configuration file, which
is described on detail in Section \ref{ug.sec.ncconf}. 
%
\subsubsection{Possible \texttt{[domain size]} section entries}
%
The \texttt{[domain size]} section specifies the model domain,
and may contain these entries:
%
\begin{center}
\begin{tabular}{|c|c|c|l|}
\hline
Name & Type & Default & Description (units) \\
\hline
\hline
\texttt{east-west} & integer & -- & Number of grid-points in the east-west
direction \\
\hline
\texttt{north-south} & integer & -- & Number of grid-points in the north-south
direction \\
\hline
\texttt{vertical} & integer & 11 & Number of model levels \\
\hline
\end{tabular}
\end{center}
%
\subsubsection{Possible \texttt{[projection]} section entries}
%
The \texttt{[projection]} section specifies the parameters of the map projection, and
may contain these elements:
%
\begin{center}
\begin{tabular}{|c|c|c|l|}
\hline
Name & Type & Default & Description (units) \\
\hline
\hline
\texttt{type} & integer & 1 & Type of projection: \\
 & & & 1) Lambert Equal Area \\
 & & & 2) Spherical polar \\
 & & & 3) Spherical stereographic (oblique) \\
 & & & 4) Spherical stereographic (equatorial) \\
\hline  
\texttt{central meridian} & real & -- & Longitide of projection centre (degrees east) \\
\hline
\texttt{parallel} & real & -- & Latitude of projection centre (degrees)\\
\hline
\texttt{false easting} & real & -- & Local $x$-coordinate of projection centre
(grid-points) \\
\hline
\texttt{false northing} & real & -- & Local $y$-coordinate of projection centre
(grid-points) \\
\hline
\texttt{standard parallel} & real & -- & Standard parallel, where the
projection \\
 & & & requires it, and where it is different to the \\
 & & & main parallel specified above. \\ 
\hline
\end{tabular}
\end{center}
%
\subsubsection{Possible \texttt{[sigma coordinates]} section entries}
%
There is only one possible entry, \texttt{file}, which gives the name a file
containg the list of sigma coordinates.
%
\subsubsection{Possible \texttt{[options]} section entries}
%
The \texttt{[options]} section specifies various ice model options, and may
contain these elements:
%
\begin{center}
    \tablefirsthead{%
    \hline
Name & Type & Default & Description (units)\\
    \hline
\hline}
  \tablehead{%
    \hline
    \multicolumn{4}{|p{0.98\textwidth}|}{\emph{\small continued from previous page}}\\
    \hline
Name & Type & Default & Description (units)\\
    \hline
\hline}
  \tabletail{%
    \hline
    \multicolumn{4}{|r|}{\emph{\small continued on next page}}\\
    \hline}
  \tablelasttail{\hline}


\begin{supertabular}{|c|c|c|p{8.5cm}|}
\texttt{whichtemp} & integer & 1 & {\raggedright
 Method of ice temperature calculation: \\
 \begin{tabular}{lp{7cm}}
  0 & Set column to surface air temperature \\
  1 & Do full temperature solution (also finds vertical velocity and apparent
  vertical velocity). \\
  2 & Set column to surface air temperature, but correct for melting-point
  effects. \\
 \end{tabular}}\\
\hline
\texttt{whichartm} & integer & 3 & {\raggedright
Method of calculation of surface air temperature:\\ 
 \begin{tabular}{lp{7cm}}
 0 & Linear function of surface elevation\\
 1 & Cubic function of distance from domain centre\\
 2 & Linear function of distance from domain centre\\
 3 & Greenland conditions (function of surface elevation and latitude) including forcing\\
 4 & Antarctic conditions (sea-level air temperature -- function of position)\\
 5 & Uniform temperature, zero range (temperature set in \texttt{cons} namelist) \\
 6 & Uniform temperature, corrected for height, zero range.\\
 7 & Use large-scale temperature and range.\\
 \end{tabular}}
\\
\hline
\texttt{whichthck} & integer & 4 & {\raggedright
Source of initial conditions: \\
\begin{tabular}{lp{7cm}}
1 & Read from file\\
2 & Set equal to one time-step of net accumulation (where positive)\\
3 & Stepped, linear function of distance from domain centre\\
4 & Read from file\\
5-7& Unknown\\
\end{tabular}}\\
\hline
\texttt{whichflwa} & integer & 0 & {\raggedright Method for calculating flow factor $A$:\\
\begin{tabular}{lp{7cm}}
0 & \emph{Patterson and Budd} relationship\\
1 & \emph{Patterson and Budd} relationship, with temperature set to $-10^{\circ}\mathrm{C}$ \\
2 & Set equal to $1\times 10^{-16}\,\mathrm{yr}^{-1}\,\mathrm{Pa}^{-n}$\\
\end{tabular}}\\
\hline
\texttt{whichisot} & integer & 1 & {\raggedright
Bedrock elevation: \\
\begin{tabular}{lp{7cm}}
0 & Fixed at input values\\
1 & Local function of ice loading history (ODE)\\
2 & Local function of ice loading history (ODE) with flexure\\
\end{tabular}}\\
\hline 
\texttt{whichslip} & integer & 4 & {\raggedright
Horizontal bed velocity: \\
\begin{tabular}{lp{7cm}}
0 & Linear function of gravitational driving stress\\
1-3 & Unknown\\
4 & Set to zero everywhere\\
\end{tabular}}\\
\hline
\texttt{whichbwat} & integer & 2 &{\raggedright
 Basal water depth: \\
\begin{tabular}{lp{7cm}}
0 & Calculated from local basal water balance\\
1 & as (0), including constant horizontal flow\\
2 & Set to zero everywhere\\
\end{tabular}}\\
\hline
\texttt{whichmarn} & integer & 0 &{\raggedright
 Ice thickness: \\
\begin{tabular}{lp{7cm}}
0 & Set thickness to zero if relaxed bedrock is more than certain water depth (??) \\
1 &  Set thickness to zero if floating \\
2 &  No action \\
\end{tabular}}\\
\hline
\texttt{whichbtrc} & integer & 1 & {\raggedright
Basal slip coefficient: \\
\begin{tabular}{lp{7cm}}
0 & \texttt{tanh} function of basal water depth\\
1 &  Set equal to zero everywhere\\
\end{tabular}}\\
\hline
\texttt{whichacab} & integer & 2 &{\raggedright
 Net accumulation: \\
\begin{tabular}{lp{7cm}}
0 & EISMINT moving margin \\
1 & PDD mass-balance model [recommended] \\
2 & Accumulation only\\
\end{tabular}}\\
\hline
\texttt{whichevol} & integer & 0 & {\raggedright
Thickness evolution method:\\
\begin{tabular}{lp{7cm}}
0 & Pseudo-diffusion approach \\
2 & Diffusion approach (also calculates velocities)\\
\end{tabular}}\\
\hline 
\texttt{whichwvel} & integer & 0 & Vertical velocities: \\
 & & & 0) Usual vertical integration \\
 & & & 1) Vertical integration constrained so that \\
 & & & upper kinematic B.C. obeyed \\
\hline 
\texttt{whichmask} & integer & 0 & ?? \\
\hline
\texttt{whichprecip} & integer & 0 & {\raggedright
Source of precipitation:\\
\begin{tabular}{lp{7cm}}
0 & Uniform precipitation rate (set internally at present)\\
1 & Use large-scale precipitation rate\\
2 & Use parameterization of \emph{Roe and Lindzen}\\
\end{tabular}}\\
\hline
\end{supertabular}
\end{center}
%
\subsubsection{Possible \texttt{[timesteps]} section entries}
%
These entries specify the time-steps of the various model components as a
multiple of the main time-step.
\begin{center}
\begin{tabular}{|c|c|c|l|}
\hline
Name & Type & Default & Description (units)\\
\hline
\hline
\texttt{temperature} & real & 1 & Length of temperature time-step as \\
 & & & a multiple of main timestep \\
\hline
\texttt{velocity}    & real & 1 & Length of velocity time-step as \\
 & & & a multiple of main time-step\\
\hline
\texttt{isostasy}    & real & 1 & Length of isostasy time-step as \\
 & & & a multiple of main time-step\\
\hline
\end{tabular}
\end{center}
%
\subsubsection{Possible \texttt{[grid-lengths]} section entries}
%
These entries specify the dimensions of the grid-boxes used in the model instance.
\begin{center}
\begin{tabular}{|c|c|c|l|}
\hline
Name & Type & Default & Description (units)\\
\hline
\hline
\texttt{east-west} & real & -- & Horizontal grid spacing in east-west direction (m)
\\
\hline
\texttt{north-south} & real & -- & Horizontal grid spacing in north-south direction (m) \\
\hline
\end{tabular}
\end{center}
%
\subsubsection{Possible \texttt{[parameters]} section entries}
%
\begin{center}
\begin{tabular}{|c|c|c|l|}
\hline
Name & Type & Default & Description (units)\\
\hline
\hline
\texttt{geot} & real & $-5\times 10^{-2}$ & Geothermal heat flux
($\mathrm{Wm}^{-2}$) \\
\hline
\texttt{fiddle} & real & 3.0 & Multiplier for flow factor \\
\hline
\texttt{airt} & real & (-3.15,-0.01) & Air temperature parameterization
factors \\
(2 elements)& & & (K, $\mathrm{K}\,\mathrm{km}^{-3}$) \\
\hline
\texttt{nmsb} & real & (0.5, $1.05\times 10^{-5}$,  & Net accumulation
factors used in \\
(3 elements) & & $4.5\times 10^{5}$) & combination with $\mathtt{whichthck}=2,3$ \\
 & & & ($\mathrm{m}\,\mathrm{yr}^{-1}$, $\mathrm{yr}^{-1}$, m) \\
\hline
\texttt{hydtim} & real & 1000 & 1) Basal hydrology time constant (When \\
 & & & $\mathtt{whichbwat}=0$) (yr)\\
 & & & 2) Basal hydrology advection velocity (When \\
 & & & $\mathtt{whichbwat}=1$) ($\mathrm{m}\,\mathrm{yr}^{-1}$)\\
\hline
\texttt{isotim} & real & 3000 & Isostasy time-constant (used in combination \\
 & & & with \texttt{whichisot} (yr) \\
\hline 
\texttt{bpar}  & real & (2.0, 10.0, 10.0, & Basal traction factors (used in \\
 (5 elements)& & 0.0, 1.0) & combination with \texttt{whichbtrc}). These describe \\
 & & & the form of the $B=\tanh(W)$ function: \\
 & & & (1) Width of tanh curve \\
 & & & (2) $W$ at midpoint of tanh curve (m) \\
 & & & (3) $B$ minimum ($\mathrm{m}\,\mathrm{yr}^{-1}\,\mathrm{Pa}^{-1}$) \\
 & & & (4) $B$ maximum ($\mathrm{m}\,\mathrm{yr}^{-1}\,\mathrm{Pa}^{-1}$) \\
 & & & (5) multiplier for marine sediments \\
\hline
\texttt{thklim} & real & 100 & Lower limit for 3D ice calculations (m?) \\
\hline
\texttt{mlimit} & real & -- & ? \\
\hline
\end{tabular}
\end{center}
%
\subsubsection{Possible \texttt{dats} namelist entries}
%
\begin{center}
\begin{tabular}{|c|c|c|l|}
\hline
Name & Type & Default & Description (units)\\
\hline
\hline
& & & \\
\hline
\end{tabular}
\end{center}
%
\subsubsection{Possible \texttt{cons} namelist entries}
%
\begin{center}
\begin{tabular}{|c|c|c|l|}
\hline
Name & Type & Default & Description (units)\\
\hline
\hline
\texttt{lapse\_rate} & real & -8.0 & Lapse rate used when adjusting \\
 & & & air temperature for elevation ($\mathrm{K}\,\mathrm{km}^{-1}$) \\
\hline
\texttt{precip\_rate} & real & 0.5 & Uniform precipitation rate,
\\
 & & & used in conjuction with \texttt{whichprecip}=0 \\
 & & & ($\mathrm{m}\,\mathrm{yr}^{-1}$ water equivalent) \\
\hline
\texttt{air\_temp} & real & -20.0 & Uniform surface air temperature, \\
 & & & used in conjunction with \texttt{whichsurftemp}=0,1 ($^{\circ}\mathrm{C}$) \\
\hline
\texttt{albedo} & real & 0.4 & Ice albedo \\
\hline
\end{tabular}
\end{center}
%
\subsection{Possible \texttt{forc} namelist entries}
%
\begin{center}
\begin{tabular}{|c|c|c|l|}
\hline
Name & Type & Default & Description (units)\\
\hline
\hline
\texttt{trun} & real & -- & Length of model run (years).\\
 & & & {\bf N.B.} This variable {\em does not} control the length of \\
 & & & the model run --- it is used in the case when \texttt{whichartm}=3 \\
 & & & to tell the forcing initialisation how long the run is going to be. \\
\hline
\end{tabular}
\end{center}