\section{GLIMMER in practice -- an example}

\subsection{Initialising and calling}

The easiest way to learn how GLIMMER is used is by way of an
example. We assume that the GLIMMER code has been installed alongside the
climate model code, and can be compiled and linked successfully. Details of
how to achieve this may be found in the \texttt{COMPILE} file in the top-level
GLIMMER directory. 

Typically, GLIMMER will be called from the main program body of a
climate model. To make this possible, the compiler needs to be told to use the
GLIMMER code. Use statements appear at the very beginning of f90 program
units, before even \texttt{implicit none}:
%
\begin{verbatim}
  use glimmer_main
\end{verbatim}
%
The next task is to declare a variable of type \texttt{glimmer\_params}, which
holds everything relating to the model, including any number of ice-sheet
instances:
%
\begin{verbatim}
  type(glimmer_params) :: ice_sheet
\end{verbatim}
%
Before the ice-sheet model may be called from the climate model, it must be
initialised. This is done with the following subroutine call:
%
\begin{verbatim}
  call initialise_glimmer(ice_sheet,lats,lons,paramfile)
\end{verbatim}
%
In this call, the arguments are as follows:
%
\begin{itemize}
\item \texttt{ice\_sheet} is the variable of type \texttt{glimmer\_params}
 defined above;
\item \texttt{lats} and \texttt{lons} are one-dimensional arrays giving the
  locations of the global grid-points in latitude and longitude, respectively; 
\item \texttt{paramfile} is the file name of the top-level GLIMMER parameter
  namelist.
\end{itemize}
%
The contents of the namelist files will be dealt with later. Having
initialised the model, it may now be called as part of the main climate
model time-step loop:
%
\begin{verbatim}
    call glimmer(ice_sheet,time,temp,precip,zonwind,merwind,orog)
\end{verbatim} 
%
The arguments given in this example are the compulsory ones only; a number of
optional arguments may be specified -- these are detailed in the reference
section below. The compulsory arguments are:
%
\begin{itemize}
\item \texttt{ice\_sheet} is the variable of type \texttt{glimmer\_params}
 defined above;
\item \texttt{time} is the current model time, in years;
\item \texttt{temp} is the daily mean $2\,\mathrm{m}$ global air temperature field, in
  $^{\circ}\mathrm{C}$;
\item \texttt{precip} is the global daily precipitation fields,
  in $\mathrm{mm}/\mathrm{day}$ (water equivalent, making no distinction
  between rain, snow, etc.);
\item \texttt{zonwind} and \texttt{merwind} are the daily mean global zonal and
  meridional components of the $10\,\mathrm{m}$ wind field, in
  $\mathrm{ms}^{-1}$;
\item \texttt{orog} is the global orography field, in $\mathrm{m}$.
\end{itemize}
%
For the positive degree-day mass-balance routine, which is currently the only
mass-balance model included with GLIMMER, the daily quantities given above are
necessary, and, as such, GLIMMER should be called once per day. With the
energy and mass-balance model currently being developed, hourly calls will be
necessary. 
%
\subsection{Finishing off}
%
After the desired number of time-steps have been run, GLIMMER may have some
tidying up to do. To accomplish this, the subroutine \texttt{end\_glimmer}
must be called:
%
\begin{verbatim}
  call end_glimmer(ice_sheet)
\end{verbatim}
%

\subsection{Parameter Files}
\subsubsection{Namelists}
%
Global parameters, applicable to all instances of the ice model are contained
in the file specified by \texttt{paramfile} in the call to
\texttt{initialise\_glimmer}. An example namelist file is given below (it is
the file \texttt{example.glp}):
%
\begin{verbatim}
&timesteps
tinc = 1.0/
&file_paras
ninst=1/
g_land.gln
\end{verbatim}
%
In FORTRAN, a namelist begins with \texttt{\&}, followed by that namelist's
label, and a list of variable names and their values, separated by either
commas or newlines. Note that not all the members of a namelist need be
present, nor do they have to appear in the order declared in the code. 
The namelist ends with a \texttt{/} character. As you can see, the GLIMMER
parameter namelist file contains two namelists, each containing a single
variable, and ends with a field of plain text (in this case, `\texttt{g\_land.gln}').

The first namelist, \texttt{timesteps} contains the main ice model time-step,
\texttt{tinc}, in years. The second namelist, \texttt{file\_paras} contains
the number of ice model instances, \texttt{ninst}. After the two namelists are
the filenames of the namelists of parameters for each ice model
instance. Thus, there should be a number of filenames equal to the value of \texttt{ninst}.

For each instance of the model, a namelist file with instance-specific
parameters must be supplied. The names of these files are given in the main
GLIMMER namelist, as described above. The namelist file for an individual
instance is fairly long, so only a summary of the different namelists
contained in it is given below. For an example this kind of file, see
\texttt{g\_land.gln} in the Inputs directory.

Namelists in an instance-specific file:
%
\begin{itemize}
\item The first line of the instance-specific file contains the name of the configuration file controlling netCDF I/O. In the case of the example file
  \texttt{g\_land.gln}, the netCDF control file is \texttt{g\_land}, see Section \ref{ug.sec.ncconf}.
\item \texttt{sizs} Model grid parameters -- number of grid points in $x$ and
  $y$ dimensions, and number of levels in the vertical.
\item \texttt{prj} Details of the projection used in the instance -- type of
  projection, and necessary parameters.
\item Next, there follows a field of plain text, specifying the name of a
  sigma-coordinates file (see below)
\item \texttt{opts} Flags specifying various model integration options, such
  as which of various schemes to use, etc.
\item \texttt{nums} Parameters relevant to the model numerics -- time-steps and
  things.
\item \texttt{pars} Physical parameters for the model, such as isostatic
  relaxation timescale.
\item \texttt{cons} Constants used by some parts of the model.
\item \texttt{forc} Parameters used by the forcing which may be used to run
  the model in stand-alone mode.
\end{itemize}
%
The contents of the namelist files is likely to be altered in the future, and
they may also be replaced by XML files.
%
\subsubsection{The Sigma coordinate file}
%
The name of the file containg the sigma variable is specified in
the instance-specific namelist file. This file consists of a
list of numbers in ascending order, between 0.0 and 1.0, specifying the
heights of the model levels in sigma space. The number of entries must be the
same as the number of model levels specified. An example file is
\texttt{g\_land.gls}.

\subsubsection{Grid I/O control files}
The model requires at least one field as input -- the height of the bedrock
topography (including bathymetry, if appropriate). If the bedrock is not in a relaxed state, then
the height of the relaxed topography must also be supplied. The bedrock is assumed to be relaxed 
if no relaxed topography field is supplied. Furthermore, the uppser surface of the ice sheet at the 
start of the run can be specified.

Each model instance can save specific variables to different netCDF files with different time intervals.

\subsection{Viewing the output}
%
To view output from glimmer, a set of Matlab routines is supplied. Assuming
that you have Matlab configured to search for the \texttt{.m} files in the
right place, the results may be loaded into Matlab using one of the following
functions:
%
\begin{itemize}
\item \texttt{glim0d(pathname,stem);} Returns an array of diagnostic
  data. \emph{Not ready for use yet\ldots}
\item \texttt{glim2d(time,pathname,stem);} Returns a structure containing
  two-dimensional data. For example use, see below.
\item \texttt{glim3d(time,pathname,stem);} Returns a structure containing
  three-dimensional data. \emph{Not ready for use yet\ldots}
\end{itemize}
%
In these function calls:
\begin{itemize}
\item \texttt{pathname} specifies the location of the file
to be read. If the file is in the current working directory, then
\texttt{pathname} should be set to \texttt{'.'}
\item \texttt{stem} is the output filename stem specified in the
  instance-specific namelist file (see above). In the case of the example
  given above, \texttt{stem} would be set to \texttt{'g\_land'}
\item \texttt{time} is the time in years for which data is required.
\end{itemize}
%
For example, the 1000-year two-dimensional fields output by the model when initialised
by the supplied example namelist files may be read into a Matlab structure
\texttt{a} with the command:
%
\begin{verbatim}
>> a=glim2d(1000,'.','g_land');
\end{verbatim}
%
If the function finds corresponding data, it outputs a list of the fields
obtained. A typical list looks like this:
%
\begin{verbatim}
got uq - scale 1000000
got vq - scale 1000000
got df - scale 100000000
got bx - scale 25
got bu - scale 500
got bv - scale 500
got th - scale 2000
got up - scale 2000
got lw - scale 2000
got to - scale 2000
got ac - scale 5
got bm - scale 5
got bw - scale 2000
got at - scale 1
got bt - scale 1
got ar - scale 1
got pc - scale 5
got ab - scale 5
got ds - scale 5
\end{verbatim}
% 
You don't need to worry what the `scale' field means. What
matters is the list of two-letter codes for the model fields now residing in
memory. A full list is given in the reference section below, but in this
example we'll consider \texttt{up}, which is the upper surface height --
i.e. the elevation of the ice surface or the topography where no ice is
present, taking into account the presence of ocean. To plot this upper
surface, either do so with the \texttt{surf} function
%
\begin{verbatim}
>> surf(a.up);
\end{verbatim}
%
or with one of the contour plotting functions:
%
\begin{verbatim}
>> contourf(a.up,20);
\end{verbatim}
%
And that's all there is to it.
%
\subsection{Two-dimensional output files -- Matlab two-letter codes}
%
The two letter variable names contained in the structure returned by
\texttt{glim2d} have the following meanings:
%
\begin{center}
\begin{tabular}{|c|l|c|}
\hline
Code & Description & units \\
\hline
\hline
\texttt{uq} & flux in $x$ direction & $\mathrm{m}^2\mathrm{yr}^{-1}$ \\
\texttt{vq} & flux in $y$ direction & $\mathrm{m}^2\mathrm{yr}^{-1}$ \\
\texttt{df} & apparent diffusivity & $\mathrm{m}^2\mathrm{yr}^{-1}$ \\
\texttt{bx} & basal slip coefficient & $\mathrm{m\,Pa^{-1}\,yr^{-1}}$ \\
\texttt{bu} & basal slip velocity in $x$ direction & $\mathrm{m\,yr^{-1}}$\\
\texttt{bv} & basal slip velocity in $y$ direction & $\mathrm{m\,yr^{-1}}$\\
\texttt{th} & thickness & m \\
\texttt{up} & ice upper surface elevation & m \\
\texttt{lw} & ice lower surface elevation & m \\
\texttt{to} & bedrock topography & m \\
\texttt{ac} & accumulation-ablation rate & $\mathrm{m\,yr^{-1}}$ \\
\texttt{bm} & basal melt rate & $\mathrm{m\,yr^{-1}}$ \\
\texttt{bw} & basal water depth & m \\
\texttt{at} & annual mean air temperature & $^{\circ}\mathrm{C}$ \\
\texttt{bt} & basal ice temperature & $^{\circ}\mathrm{C}$ \\
\texttt{ar} & annual air temperature range & $^{\circ}\mathrm{C}$ \\
\texttt{pc} & precipitation & $\mathrm{m\,yr^{-1}}$ \\
\texttt{ab} & ablation rate & $\mathrm{m\,yr^{-1}}$ \\
\texttt{ds} & rate of upper ice surface elevation change &
$\mathrm{m\,yr^{-1}}$ \\
\hline
\end{tabular}
\end{center}
%
\subsection{Restarts}
%
GLIMMER provides two routines to handle restarts,
\texttt{glimmer\_write\_restart}, and \texttt{glimmer\_read\_restart}. The
former writes the entire model state to a single file, while the latter will
restore the model state from a previously created
file. For example, \texttt{glimmer\_write\_restart} may be called as follows:
%
\begin{verbatim}
  call glimmer_write_restart(ice_sheet,25,'ice_sheet.restart')
\end{verbatim}
%
Here, \texttt{ice\_sheet} is the GLIMMER parameter variable refered to
previously, \texttt{25} is the logical file-unit to use, and
\texttt{'ice\_sheet.restart'} is the filename of the restart file. This
subroutine call may be made at any point, regardless of whether it is intended
to halt the integration imminently, or not. In order to recover the model
state, the following call to \texttt{glimmer\_read\_restart} would be made:
%
\begin{verbatim}
  call glimmer_read_restart(ice_sheet,25,'ice_sheet.restart')
\end{verbatim}
%
The arguments are the same as for the previous call. When restarting from a
file like this, it is not necessary to make a call to
\texttt{initialise\_glimmer}. Note also that there is no alternative restart mechanism
provided within the normal \texttt{glimmer} subroutine call -- all restarts
must be called explicitly.

Note also that \texttt{glimmer\_read\_restart} may not be called if
\texttt{initialise\_glimmer} has been called already. This is because there is
currently no mechanism for \texttt{glimmer\_read\_restart} to know whether the
relevant model arrays have already been allocated. If they have, and
\texttt{glimmer\_read\_restart} tries to reallocate them, a fatal
run-time error will probably be generated. It is hoped to address this problem
in a future release.
%