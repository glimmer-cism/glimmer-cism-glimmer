
\section{Physics documentation}

\subsection{Ice temperature evolution routines}

\subsubsection{Summary}
Call structure (filenames in brackets).
\begin{itemize}
    \item subroutine testinisthk [glimmer\_setup] and
    \item subroutine glimmer\_i\_tstep [glimmer\_object] call
    \item subroutine timeevoltemp [glimmer\_temp] calls
    \item subroutine calcartm [glimmer\_temp] and
    \item subroutine timeders [glimmer\_thck] and
    \item subroutine gridwvel [glimmer\_velo] and
    \item subroutine wvelintg [glimmer\_velo] and
    \item subroutine chckwvel [glimmer\_velo] and
    \item subroutine finddisp [glimmer\_temp] and
    \item subroutine hadvall [glimmer\_temp] and
    \item subroutine hadvpnt [glimmer\_temp] and
    \item subroutine findvtri [glimmer\_temp] and
    \item subroutine tridag [glimmer\_temp] and
    \item subroutine corrpmpt [glimmer\_temp] and
    \item subroutine swapbndt [glimmer\_temp] and
    \item subroutine calcbmlt [glimmer\_temp] and
    \item subroutine calcflwa [glimmer\_temp]
\end{itemize}

\noindent Modules used.
\begin{itemize}
    \item
\end{itemize}

\subsubsection{Introduction}
The section describes the routines that are concerned with
calculating the three-dimensional distribution of temperature
within the ice mass.  They can be broken down into five groups.
\begin{itemize}
    \item determining air temperature (upper boundary
    condition) [\texttt{calcartm}];
    \item determining vertical velocity field from existing
    horizontal velocity fields (normally only needed if temperature is being calculated) [\texttt{wvelintg}, chckwvel];
    \item routines associated with vertical grid coordinate
    system [\texttt{gridwvel}, \texttt{timeders}];
    \item the main temperature solver [\texttt{finddisp, hadvall, hadvpnt, findvtri, tridag, corrpmpt, swapbndt}];
    \item ancillary calculations that only make sense if temperature is being calculated
    [\texttt{calcbmlt}, \texttt{calcflwa}].
\end{itemize}

The basic quantity returned is a three-dimensional grid of
temperature in $\circ^{-1}$C (uncorrected for variations in
pressure melting point and unscaled).  Temperature is held in the
array \texttt{temp} and will be referred to here using the symbol
$T$.

In addition to temperature a number of other quantities are
calculated by these routines.  They include: basal melt rate ($m$
\texttt{bmlt} m yr$^{-1}$ scaled using \texttt{thk0/tim0}); basal
water depth ($W$ \texttt{bwat} m scaled using \texttt{thk0});
vertical velocity ($w$ \texttt{wvel} m yr$^{-1}$ scaled using
\texttt{thk0/tim0}); vertical velocity of numerical grid ($w_0$
\texttt{wgrd} m yr$^{-1}$ scaled using \texttt{thk0/tim0}); Glen's
A ($A$ \texttt{flwa} Pa$^{-3}$ yr$^{-1}$ scaled using
\texttt{vis0}); air temperature ($T_a$ $\circ^{-1}$C unscaled).
All scales are held in the module \texttt{paramets} in
\textbf{\texttt{glimmer\_paramets}}.

Three options are currently available for calculating $T$. The
particular option chosen is controlled by the input parameter
\texttt{whichtemp} (\texttt{gln} file).

\begin{description}
    \item[0] Set whole column to the appropriate surface air temperature ($T_a$).
    \item[1] This option is the main solver that determines temperature
    at the new time step from the appropriate three-dimensional
    advection-diffusion equation.
    \item[2] Set the upper surface temperature to $T_a$ and do a linear
    interpolation from this value to 0 $^\circ$C at the lower
    surface. Check for pressure melting and adjust any
    temperatures that are above melting point.
\end{description}

The subroutine \texttt{timeevoltemp} controls calculation of the
$T$ etc. It is called in the main time loop in
\textbf{\texttt{glimmer\_object}} and resides in
\textbf{\texttt{glimmer\_temp}}.

\subsection{Mass balance routines}

\subsubsection{Summary}
Call structure (filenames in brackets).
\begin{itemize}
    \item subroutine testinisthk [glimmer\_setup] and
    \item subroutine glimmer\_i\_tstep [glimmer\_object] call
    \item subroutine calcacab [glimmer\_thck] calls
    \item subroutine masbgrn [glimmer\_degd]
\end{itemize}

\noindent Modules used.
\begin{itemize}
    \item glimmer\_degd [glimmer\_degd]
    \item paramets [glimmer\_paramets]
\end{itemize}

\subsubsection{Introduction} The model contains a variety of surface
mass balances schemes. The basic quantity returned is a
two-dimensional grid of surface mass balance in m yr$^{-1}$ ice
equivalents (positive for net annual accumulation and negative for
net annual ablation) scaled using the quantity \texttt{acc0} (see
module \texttt{paramets} in \textbf{\texttt{glimmer\_paramets}}).
The particular option chosen is controlled by the input parameter
\texttt{whichacab} (\texttt{gln} file).

Surface mass balance is held in the array \texttt{acab} and will
be referred to here using the symbol $b$. Three options are
currently available for calculating $b$.

\begin{description}
    \item[0] Set up radial mass balance field.  This option is
    primarily for EISMINT moving-margin benchmarks.  It uses the input
    parameter \texttt{nmsb(1:3)} (\texttt{gln} file).
    \item[1] This option is appropriate for Greenland calculations
    and does a day-degree calculation for mass balance.
    \item[2] Simply sets the mass balance equal to the
    precipitation field.  This option is suitable for modelling
    Antarctica or the EISMINT fixed-margin benchmark.
\end{description}

It is hoped (July 2004) to add a surface energy-balance option (3)
this year.

The subroutine \texttt{calcacab} controls calculation of the $b$.
It is called in the main time loop in
\textbf{\texttt{glimmer\_object}} and resides in
\textbf{\texttt{glimmer\_thck}}.

\subsubsection{EISMINT code (option 0)}
A radial mass balance is created which is centred on the middle of
the model grid
\begin{equation}\label{eismintb}
    b(x,y) = \min(b_1,b_2\times(b_3-d))
\end{equation}
where $d$ is the distance in m from the middle of the grid and
$b_1$ is in m yr$^{-1}$, $b_2$ is in yr$^{-1}$ and $b_3$ is in m
(all have been scaled by the time they reach this subroutine).
Details of the EISMINT moving-margin test can be found in
\emph{Huybrechts and others} [1996] and the values for the $b$
parameters are: 0.5 m yr$^{-1}$, 10$^{-5}$ yr$^{-1}$ and 4.5
$\times 10^5$ m.

\paragraph{References}

Huybrechts and others (1996) The EISMINT benchmarks for testing
ice-sheet models, \emph{Annals of Glaciology} \textbf{23}, 1-12.

\subsubsection{Greenland code (option 1)}
This is the most complicated option to date.  It calls subroutine
\texttt{masbgrn}, which resides in
\textbf{\texttt{glimmer\_degd}}.

This option follows the EISMINT Greenland parameterization.  It
contains two basic components.  The first is a day-degree
calculation of ablation, and the second is an estimation of
refreezing.

Inputs are: mean annual air temperature (\texttt{artm}); annual
air temperature half range (\texttt{arng}); precipitation
(\texttt{prcp}); and latitude (\texttt{lati}).  All on
two-dimensional, spatial grids. Temperatures and latitude are not
scaled, while precipitation is assumed to have already been
scaled.  Latitude is only used as a mask and calculation are
performed only if it is non-zero (may need to look at this).

In addition to outputting \texttt{acab}, the routines output the
calculated ablation \texttt{ablt} (also scaled).  Note that any
corrections (for instance to account for changing altitude, air
temperature and atmospheric moisture) to the precipitation field
are delt with elsewhere (see \texttt{\textbf{glimmer\_mbal}}).

\paragraph{Day-degree calculation}
The parameters used in these calculations are set in
\texttt{glimmer\_pddcalc} in \textbf{\texttt{glimmer\_modu}}.

The fairly large amount of computation required here means that a
look-up table approach has been implemented.  If this is the first
time that the subroutine has been used then the look-up table is
constructed in subroutine \texttt{pddtabgrn}.

The table has two dimensions: mean annual air temperature ($T_a$)
(as the second index) and annual air temperature half range (i.e.,
from July's mean to the annual mean $\Delta T_a$) (as the first
index).  Following \emph{Huybrechts and others} [1991], daily air
temperatures ($T_a^\prime$) are assumed to follow a sinusoidal
cycle
\begin{equation}
    T_a^\prime = T_a + \Delta T_a \cos \left( \frac{2 \pi t}{A}
    \right) + \textbf{R}(0,\sigma)
\end{equation}
where $A$ is the period of a year and $R$ is a random fluctuation
drawn from a normal distribution with mean 0 $^\circ$C and
standard deviation $\sigma$ $^\circ$C. \emph{Huybrechts and
others} [1991] indicate that the number of positive degree days
(D, $^\circ$C days) for this temperature series can be evaluated
as
\begin{equation}\label{pdd}
    D = \frac{1}{\sigma \sqrt{2 \pi}}
    \int\limits_0^A
    \int\limits_0^{T_a^\prime+2.5\sigma}
    T_a \times \exp \left( \frac{-(T_a-T_a^\prime)^2}{2 \sigma^2} \right) dT
    dt
\end{equation}
where $t$ is time.  The table is completed by evaluating this integral using a
public-domain algorithm (Romberg integration), by {\it Bauer} [1961]. The
inner and outer integrals are coded as two subroutines
(\texttt{inner\_integral} and \texttt{pdd\_integrand}), which call the Romburg
integration recursively.

The main parameter needed is the assumed standard deviation of
daily air temperatures, which is set in
\textbf{\texttt{glimmer\_pddcalc}} (typically 5 $^\circ $C).  The
module also contains information on the size and extent (in terms
of $T_a$ and $\Delta T_a$) of the table to be constructed.

The table has now been constructed and we can enter the main loop
in \texttt{masbgrn}, which visits each cell in the numerical grid
in turn. Note that the variable \texttt{lati} is used as a mask
and calculation are made only for non-zero values of \texttt{lati}
(both \texttt{ablt} and \texttt{acab} are otherwise set to zero).

The positive-degree days are then looked up in the table just
described (as a function of $T_a$ and $\Delta T_a$). We take care
to check that this look up is in done within the bounds of the
table.  The final value of $P$ is determined using bi-linear
interpolation given the four nearest entries in the table to the
actual values of $T_a$ and $\Delta T_a$.

The remainder of the loop completes the calculation of the
ablation and accumulation given this value for $P$.

\paragraph{Mass balance calculation} The parameters used in these
calculations are set in \texttt{paramets} in
\textbf{\texttt{glimmer\_paramets}}. Note, in particular, the
day-degree factors have been converted from ice to water
equivalents using the ratio of densities.  They are also scaled.

We use the following symbols: $a$ is total annual ablation; $a_s$
is potential snow ablation; $b_0$ is the capacity of the snowpack
to hold meltwater by refreezing; the total number of positive
degree days ($D$); day-degree factors for snow and ice ($f_s$ and
$f_i$); and the fraction of snowfall that can be held in the
snowpack as refrozen meltwater ($W_max$).

 First,
determine the depth of superimposed ice ($b_0$) that would have to
be formed before runoff (mass loss) occurs as a constant fraction
($W_{max}$) of precipitation ($P$)
\begin{equation}
    b_0=W_{max} P.
\end{equation}
Now determine the amount of snow melt by applying a constant
day-degree factor for snow to the number of positive day-degrees
\begin{equation}
    a_s=f_s D.
\end{equation}
We now compare the potential amount of snow ablation with the
ability of the snow layer to absorb the melt.  Three cases are
possible. First, all snow melt is held within the snowpack and no
runoff occurs ($a=0$).  Second, the ability of the snowpack to
hold meltwater is exceeded but the potential snow ablation is
still less than the total amount of precipitation so that
$a=a_s-b_0$. Finally, the potential snow melt is greater than the
precipitation (amount of snow available), so that ice melt ($a_i$)
has to be considered as well.  The total ablation is therefore the
sum of snow melt (total precipitation minus meltwater held in
refreezing) and ice melt (deduct from total number of degree days,
the number of degree days needed to melt all snowfall and convert
to ice melt)
\begin{equation}
    a=a_s + a_i = P - b_0 + f_i \left( D-\frac{P}{f_s} \right).
\end{equation}
We now have a total annual ablation, and can find total net mass
balance as the difference between the total annual precipitation
and the total annual ablation.

Note that this methodology is fairly standard and stems from a
series of Greenland papers by Huybrechts, Letreguilly and Reeh in
the early 1990s.

\paragraph{References}

Bauer (1961) \emph{Comm. ACM} \textbf{4}, 255.

Huybrechts, Letreguilly and Reeh (1991) \emph{Palaeogeography,
Palaeoclimatology, Palaeoecology (Global and Planetary Change)}
\textbf{89}, 399-412.

Letreguilly, Reeh and  Huybrechts (1991) \emph{Palaeogeography,
Palaeoclimatology, Palaeoecology (Global and Planetary Change)}
\textbf{90}, 385-394.

Letreguilly, Huybrechts  and  Reeh (1991) \emph{Journal of
Glaciology} \textbf{37}, 149-157.

\subsubsection{Antarctic code (option 2)}
This option simply sets $b$ to the array \texttt{prcp}.

\subsubsection{Issues}
\begin{enumerate}
    \item Use of \texttt{lati} as a mask seems unnecessary perhaps
    use land/sea mask.
    \item Need to discuss location of precipitation calculations
    and where they are called from.
    \item Ditto for air temperature calculations.
\end{enumerate}
