\section{GLIDE}
GLIDE is the actual ice sheet model. GLIDE comprises three procedures which initialise the model, perform a single time step and finalise the model. GLIDE configuration and API are described in the following subsections.

\subsection{Configuration}
The format of the configuration files is similar to Windows \texttt{.ini} files and contains sections. Each section contains key, values pairs.
\begin{itemize}
 \item Empty lines, or lines starting with a \texttt{\#}, \texttt{;} or \texttt{!} are ignored.
 \item A new section starts with the the section name enclose with square brackets, e.g. \texttt{[grid]}.
 \item Keys are separated from their associated values by a \texttt{=} or \texttt{:}.
\end{itemize}
Sections and keys are case sensitive and may contain white space. However, the configuration parser is very simple and thus the number of spaces within a key or section name also matters. Sensible defaults are used when a specific key is not found.


\begin{center}
    \tablefirsthead{%
    \hline
    }
  \tablehead{%
    \hline
    \multicolumn{2}{|p{0.98\textwidth}|}{\emph{\small continued from previous page}}\\
    \hline
    }
  \tabletail{%
    \hline
    \multicolumn{2}{|r|}{\emph{\small continued on next page}}\\
    \hline}
  \tablelasttail{\hline}
  \begin{supertabular}{|l|p{9cm}|}
  %%%% GRID
  \hline
  \multicolumn{2}{|l|}{\texttt{[grid]}}\\
  \hline
  \multicolumn{2}{|p{0.98\textwidth}|}{Define model grid. Maybe we should make this optional and read grid specifications from input netCDF file (if present). Certainly, the input netCDF files should be checked (but presently are not) if grid specifications are compatible.}\\
  \hline
  \texttt{ewn} & (integer) number of nodes in $x$--direction\\
  \texttt{nsn} & (integer) number of nodes in $y$--direction\\
  \texttt{upn} & (integer) number of nodes in $z$--direction\\
  \texttt{dew} & (real) node spacing in $x$--direction\\
  \texttt{dns} & (real) node spacing in $y$--direction\\
  \texttt{sigma\_file} & (string) Name of file containing $\sigma$ coordinates. The $\sigma$ coordinates are calculated if no file name is given using the formula 
	$$\sigma_i=\frac{1-(x_i+1)^{-n}}{1-2^{-n}}\quad\mbox{with}\quad x_i=\frac{\sigma_i-1}{\sigma_n-1}, n=2$$ We should probably allow $n$ to be a run--time parameter.\\
  \hline
  %%%% TIME
  \hline
  \multicolumn{2}{|l|}{\texttt{[time]}}\\
  \hline
  \multicolumn{2}{|p{0.98\textwidth}|}{Configure time steps, etc. Update intervals should probably become absolute values rather than related to the main time step when we introduce variable time steps.}\\
  \hline
  \texttt{tstart} & (real) Start time of the model in years\\
  \texttt{tend} & (real) End time of the model in years\\
  \texttt{dt} & (real) size of time step in years\\
  \texttt{ntem} & (real) time step multiplier setting the ice temperature update interval\\
  \texttt{nvel} & (real) time step multiplier setting the velocity update interval\\
  \texttt{niso} & (real) time step multiplier setting the isostasy update interval\\
  \hline
  %%%% Options
  \hline
  \multicolumn{2}{|l|}{\texttt{[options]}}\\
  \hline
  \multicolumn{2}{|p{0.98\textwidth}|}{Parameters set in this section determine how various components of the ice sheet model are treated. Defaults are indicated in bold.}\\
  \hline
  \texttt{ioparams} & (string) name of file containing netCDF I/O configuration. The main configuration file is searched for I/O related sections if no file name is given (default).\\
  \texttt{temperature} & 
              \begin{tabular}[t]{cl}
               0 & isothermal\\
               {\bf 1} & full \\
	      \end{tabular}\\
  \texttt{flow\_law} & 
              \begin{tabular}[t]{cl}
              {\bf 0} & Patterson and Budd\\
              1 & Patterson and Budd (temp=-10degC)\\
              2 & const $10^{-16}$a$^{-1}$Pa$^{-n}$\\
	      \end{tabular}\\
  \texttt{isostasy} & 
              \begin{tabular}[t]{cl}
              0 & none\\
              {\bf 1} & local\\
              2 & elastic\\
	      \end{tabular}\\
  \texttt{sliding\_law} & 
              \begin{tabular}[t]{cl}
              0 & linear function of grav driving stress\\
              $[1-3]$ & unknown \\
              {\bf 4} & zero everywhere \\
	      \end{tabular}\\
  \texttt{basal\_water} & 
              \begin{tabular}[t]{cl}
              0 & local water balance\\
              1 & local water balance + const flux \\
              {\bf 2} & none\\
	      \end{tabular}\\
  \texttt{marine\_margin} & 
              \begin{tabular}[t]{cl}
              {\bf 0} & threshold\\
              1 & no ice shelf\\
              2 & ignore marine margin\\
	      \end{tabular}\\
  \texttt{slip\_coeff} & 
              \begin{tabular}[t]{cl}
              0 & $\propto$ basal water\\
              {\bf 1} & zero\\
	      \end{tabular}\\
  \texttt{stress\_calc} & 
              \begin{tabular}[t]{cl}
              0 & zeroth-order\\
              1 & first-order\\
              {\bf 2} & vertically-integrated first-order\\
              3 & none\\
	      \end{tabular}\\
  \texttt{evolution} & 
              \begin{tabular}[t]{cl}
              {\bf 0} & pseudo-diffusion\\
              1 & unknown \\
              2 & diffusion \\
	      \end{tabular}\\
  \texttt{vertical\_integration} & 
              \begin{tabular}[t]{cl}
              {\bf 0} & standard\\
              1 & obey upper BC\\
	      \end{tabular}\\
  \texttt{topo\_is\_relaxed} &  
              \begin{tabular}[t]{cp{8cm}}
              {\bf 0} & relaxed topography is read from a separate variable\\
              1 & first time slice of input topography is assumed to be relaxed\\
	      \end{tabular}\\
  \hline
  %%%%
  \hline
  \multicolumn{2}{|l|}{\texttt{[parameters]}}\\
  \hline
  \multicolumn{2}{|p{0.98\textwidth}|}{Set various parameters.}\\
  \hline
  \texttt{ice\_limit} & (real) below this limit ice is only accumulated, ice dynamics are switched on once the ice thickness is above this value.\\
  \texttt{marine\_limit} & (real) all ice is assumed lost once water depths reach this value. Note, water depth is negative. \\
  \texttt{geothermal} & (real) geothermal heat flux \\
  \texttt{flow\_factor} & (real) the flow law is enhanced with this factor \\
  \texttt{hydro\_time} & (real) basal hydrology time constant \\
  \texttt{isos\_time} & (real) isostasy time constant \\
  \texttt{basal\_tract} & (real(5)) basal traction factors \\
  \hline
  
  \end{supertabular}
\end{center}

\subsection{Initialisation}

\subsection{Time Step}

\subsection{Finalisation}