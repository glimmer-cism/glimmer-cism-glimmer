\section{GLIDE}
GLIDE is the actual ice sheet model. GLIDE comprises three procedures which initialise the model, perform a single time step and finalise the model. GLIDE configuration and API are described in the following subsections.

\subsection{Configuration}\label{ug.sec.config}
The format of the configuration files is similar to Windows \texttt{.ini} files and contains sections. Each section contains key, values pairs.
\begin{itemize}
\item Empty lines, or lines starting with a \texttt{\#}, \texttt{;} or \texttt{!} are ignored.
\item A new section starts with the the section name enclose with square brackets, e.g. \texttt{[grid]}.
\item Keys are separated from their associated values by a \texttt{=} or \texttt{:}.
\end{itemize}
Sections and keys are case sensitive and may contain white space. However, the configuration parser is very simple and thus the number of spaces within a key or section name also matters. Sensible defaults are used when a specific key is not found.


\begin{center}
  \tablefirsthead{%
    \hline
  }
  \tablehead{%
    \hline
    \multicolumn{2}{|p{0.98\textwidth}|}{\emph{\small continued from previous page}}\\
    \hline
  }
  \tabletail{%
    \hline
    \multicolumn{2}{|r|}{\emph{\small continued on next page}}\\
    \hline}
  \tablelasttail{\hline}
  \begin{supertabular}{|l|p{9cm}|}
%%%% GRID
    \hline
    \multicolumn{2}{|l|}{\texttt{[grid]}}\\
    \hline
    \multicolumn{2}{|p{0.98\textwidth}|}{Define model grid. Maybe we should make this optional and read grid specifications from input netCDF file (if present). Certainly, the input netCDF files should be checked (but presently are not) if grid specifications are compatible.}\\
    \hline
    \texttt{ewn} & (integer) number of nodes in $x$--direction\\
    \texttt{nsn} & (integer) number of nodes in $y$--direction\\
    \texttt{upn} & (integer) number of nodes in $z$--direction\\
    \texttt{dew} & (real) node spacing in $x$--direction\\
    \texttt{dns} & (real) node spacing in $y$--direction\\
    \texttt{sigma\_file} & (string) Name of file containing $\sigma$ coordinates. The $\sigma$ coordinates are calculated if no file name is given using the formula 
    $$\sigma_i=\frac{1-(x_i+1)^{-n}}{1-2^{-n}}\quad\mbox{with}\quad x_i=\frac{\sigma_i-1}{\sigma_n-1}, n=2$$ We should probably allow $n$ to be a run--time parameter.\\
    \hline
%%%% TIME
    \hline
    \multicolumn{2}{|l|}{\texttt{[time]}}\\
    \hline
    \multicolumn{2}{|p{0.98\textwidth}|}{Configure time steps, etc. Update intervals should probably become absolute values rather than related to the main time step when we introduce variable time steps.}\\
    \hline
    \texttt{tstart} & (real) Start time of the model in years\\
    \texttt{tend} & (real) End time of the model in years\\
    \texttt{dt} & (real) size of time step in years\\
    \texttt{ntem} & (real) time step multiplier setting the ice temperature update interval\\
    \texttt{nvel} & (real) time step multiplier setting the velocity update interval\\
    \texttt{niso} & (real) time step multiplier setting the isostasy update interval\\
    \hline
%%%% Options
    \hline
    \multicolumn{2}{|l|}{\texttt{[options]}}\\
    \hline
    \multicolumn{2}{|p{0.98\textwidth}|}{Parameters set in this section determine how various components of the ice sheet model are treated. Defaults are indicated in bold.}\\
    \hline
    \texttt{ioparams} & (string) name of file containing netCDF I/O configuration. The main configuration file is searched for I/O related sections if no file name is given (default).\\
    \texttt{temperature} & 
    \begin{tabular}[t]{cl}
      0 & isothermal\\
      {\bf 1} & full \\
    \end{tabular}\\
    \texttt{flow\_law} & 
    \begin{tabular}[t]{cl}
      {\bf 0} & Patterson and Budd\\
      1 & Patterson and Budd (temp=-10degC)\\
      2 & const $10^{-16}$a$^{-1}$Pa$^{-n}$\\
    \end{tabular}\\
    \texttt{isostasy} & 
    \begin{tabular}[t]{cl}
      0 & none\\
      {\bf 1} & local\\
      2 & elastic\\
    \end{tabular}\\
    \texttt{sliding\_law} & 
    \begin{tabular}[t]{cl}
      0 & linear function of grav driving stress\\
      $[1-3]$ & unknown \\
      {\bf 4} & zero everywhere \\
    \end{tabular}\\
    \texttt{basal\_water} & 
    \begin{tabular}[t]{cl}
      0 & local water balance\\
      1 & local water balance + const flux \\
      {\bf 2} & none\\
    \end{tabular}\\
    \texttt{marine\_margin} & 
    \begin{tabular}[t]{cp{8cm}}
      0 & ignore marine margin\\
      {\bf 1} & Set thickness to zero if relaxed bedrock is more than certain water depth\\
      2 & Set thickness to zero if floating\\
      3 & Lose fraction of ice when edge cell\\
    \end{tabular}\\
    \texttt{slip\_coeff} & 
    \begin{tabular}[t]{cl}
      {\bf 0} & zero \\
      1 & set to a non--zero constant everywhere\\
      2 & set constant where the ice base is melting\\
      0 & $\propto$ basal water\\
    \end{tabular}\\
    \texttt{stress\_calc} & 
    \begin{tabular}[t]{cl}
      0 & zeroth-order\\
      1 & first-order\\
      {\bf 2} & vertically-integrated first-order\\
      3 & none\\
    \end{tabular}\\
    \texttt{evolution} & 
    \begin{tabular}[t]{cl}
      {\bf 0} & pseudo-diffusion\\
      1 & unknown \\
      2 & diffusion \\
    \end{tabular}\\
    \texttt{vertical\_integration} & 
    \begin{tabular}[t]{cl}
      {\bf 0} & standard\\
      1 & obey upper BC\\
    \end{tabular}\\
    \texttt{topo\_is\_relaxed} &  
    \begin{tabular}[t]{cp{8cm}}
      {\bf 0} & relaxed topography is read from a separate variable\\
      1 & first time slice of input topography is assumed to be relaxed\\
    \end{tabular}\\
    \texttt{hotstart} &
     Hotstart the model if set to 1. This option only affects the way the initial temperature and flow factor distribution is calculated.\\
    \hline
%%%%
    \hline
    \multicolumn{2}{|l|}{\texttt{[parameters]}}\\
    \hline
    \multicolumn{2}{|p{0.98\textwidth}|}{Set various parameters.}\\
    \hline
    \texttt{ice\_limit} & (real) below this limit ice is only accumulated, ice dynamics are switched on once the ice thickness is above this value.\\
    \texttt{marine\_limit} & (real) all ice is assumed lost once water depths reach this value. Note, water depth is negative. \\
    \texttt{geothermal} & (real) geothermal heat flux \\
    \texttt{flow\_factor} & (real) the flow law is enhanced with this factor \\
    \texttt{hydro\_time} & (real) basal hydrology time constant \\
    \texttt{isos\_time} & (real) isostasy time constant \\
    \texttt{basal\_tract\_const} & constant basal traction parameter\\
    \texttt{basal\_tract} & (real(5)) basal traction factors. Basal traction is set to $B=\tanh(W)$ where the parameters
      \begin{tabular}{cp{7cm}}
       (1) & width of the $\tanh$ curve\\
       (2) & $W$ at midpoint of $\tanh$ curve [m]\\
       (3) & $B$ minimum [ma$^{-1}$Pa$^{-1}$] \\
       (4) & $B$ maximum [ma$^{-1}$Pa$^{-1}$] \\
       (5) & multiplier for marine sediments \\
     \end{tabular}\\
    \hline
    
  \end{supertabular}
\end{center}

netCDF I/O can be configured in the main configuration file or in a separate file (see \texttt{ioparams} in the \texttt{[options]} section). Any number of input and output files can be specified. Input files are processed in the same order hey occur in the configuration file, thus potentially overwriting priviously loaded fields.

\begin{center}
  \tablefirsthead{%
    \hline
  }
  \tablehead{%
    \hline
    \multicolumn{2}{|p{0.98\textwidth}|}{\emph{\small continued from previous page}}\\
    \hline
  }
  \tabletail{%
    \hline
    \multicolumn{2}{|r|}{\emph{\small continued on next page}}\\
    \hline}
  \tablelasttail{\hline}
  \begin{supertabular}{|l|p{11cm}|}
%%%% defaults
    \hline
    \multicolumn{2}{|l|}{\texttt{[default]}}\\
    \hline
    \multicolumn{2}{|p{0.98\textwidth}|}{This section contains metadata describing the experiment. Any of these parameters can be modified in the \texttt{[output]} section. The model automatically attaches a time stamp and the model version to the netCDF output file.}\\
    \hline
    \texttt{title}& Title of the experiment\\
    \texttt{institution} & Institution at which the experiment was run\\
    \texttt{references} & References that might be useful\\
    \texttt{comment} & A comment, further describing the experiment\\
    \hline
%%%% 
    \hline
    \multicolumn{2}{|l|}{\texttt{[input]}}\\
    \hline
    \multicolumn{2}{|p{0.98\textwidth}|}{Any number of input files can be specified. They are processed in the order they occur in the configuration file, potentially overriding previously loaded variables.}\\
    \hline
    \texttt{name}& The name of the netCDF file to be read. Typically netCDF files end with \texttt{.nc}.\\
    \texttt{time}& The time slice to be read from the netCDF file. The first time slice is read by default.\\
    \hline
%%%% 
    \hline
    \multicolumn{2}{|l|}{\texttt{[output]}}\\
    \hline
    \multicolumn{2}{|p{0.98\textwidth}|}{This section of the netCDF parameter file controls how often selected  variables are written to file.}\\
    \hline
    \texttt{name} & The name of the output netCDF file. Typically netCDF files end with \texttt{.nc}.\\
    \texttt{start} & Start writing to file when this time is reached.\\
    \texttt{frequency} & The time interval in years, determining how often selected variables are written to file.\\
    \texttt{variables} & List of variables to be written to file. See Appendix \ref{ug.sec.varlist} for a list of known variables. Names should be separated by at least one space. The variable names are case sensitive. Append \texttt{\_spot} to the variable name to store variable at specified locations. Variable \texttt{hot} selects all variables necessary for a hotstart.\\
    \texttt{numspot} & The number of spot locations\\
    \texttt{spotx} & List of $x$--indecies specifying spot locations.\\
    \texttt{spoty} & List of $y$--indecies specifying spot locations.\\   
    \hline
  \end{supertabular}
\end{center}

\subsection{GLIDE API}
This Section describes a very basic application using GLIDE. See \texttt{simple\_glide.f90} for a very basic ice sheet driver using the API described below.
\subsubsection{Reading Configuration}
The main configuration file, described in \ref{ug.sec.config}, is stored in a derived type defined in \texttt{glimmer\_config.f90}:
\begin{verbatim}
type(ConfigSection), pointer :: config
\end{verbatim}
The configuration file is read using
\begin{verbatim}
call ConfigRead(fname,config)
\end{verbatim}
where \texttt{fname} is the name of the configuration file and \texttt{config} the previously defined pointer to the derived type holding the configuration.

\subsubsection{Initialisation}
All data related to the ice sheet model is stored in a derived type defined:
\begin{verbatim}
type(glide_global_type) :: model
\end{verbatim}
This means multiple ice sheet instances can be run by a single program. It is not clear if GLIDE is thread--safe, however, we plan to investigate this and see if multiple GLIDE instances can be run in parallel using OpenMP or MPI.

Memory is allocated and data initialised by the initialisation procedure:
\begin{verbatim}
call glide_initialise(model,config)
\end{verbatim}
where \texttt{model} is the derived type holding the ice sheet model instance and \texttt{config} is a pointer to the configuration.

\subsubsection{Time Step}
The model is stepped forward in time with 
\begin{verbatim}
call glide_tstep(model,time)
\end{verbatim}
where \texttt{model} is the derived type holding the ice sheet model instance and \texttt{time} the current model time. It is up to the user to provide the surface mass balance and surface temperature by setting the variables \texttt{model\%climate\%acab(:,:)} (mass balance) and \texttt{model\%climate\%artm(:,:)}.

\subsubsection{Finalisation}
Once the model has finished running, data are deallocated and open files closed with a call to
\begin{verbatim}
call glide_finalise(model)
\end{verbatim}